\documentclass[../main.tex]{subfiles}
%\graphicspath{{\subfix{{../../images/}}}}
% !TeX root = ../main.tex
\begin{document}
	
	\subsection{Volume dependency}
	
	The results of \autoref{sec:volstudy_rnd} showed how the $\delta$-ferrite influences the lifetime of the microstructure. 
	Understanding how the influence arises must first be described, why the purely martensitic structures perform better than the purely ferritic ones.
	
	This observation can be explained by comparing the material parameters of $\delta$-ferrite and martensite. 
	The $\delta$-ferriteis the more ductile and softer phase; critical shear stress in $\delta$-ferrite is lower than in martensite.
	Plastic strains occur at lower stresses, and thus, more plastic strains arise in the ferritic microstructure. That more plastic strains arise in the $\delta$-ferrite is recognizable by the higher \gls{fip} (Figure \ref{plt:strain_single_comp_fmax}).
	On the other hand, $\delta$-ferrite can absorb more plastic strain until a crack is initiated since the critical \gls{fip} is higher for the $\delta$-ferrite than for the martensite \autoref{tab:matparams}.
	Nonetheless, compared to the 60HRC martensite with the $\delta$-ferrite, the martensite experiences minor damage relative to its critical value and thus has better fatigue properties.
	However, this is not generally true in comparison to the 37HRC martensite, as the lifetime is shorter than the ferrite's lifetime at minor strains. At minor strains, both pure microstructures have similar amounts of plastic strain. However, due to the difference in the critical \gls{fip}, the ferrite has a better lifetime. With higher strains,  the plastic strains in the ferrite get significantly more, so that the martensite performs better in terms of lifetime.\\
	
	In the case of multiphase microstructures, the different properties of the phases increase the inhomogeneity of the microstructure, which affects the lifetime. As seen in Figure \ref{plt:phases_lifetime_strain} the lifetimes are reduced compared to the pure microstructures. The extent of the change depends not only on the load level but also on the volume distribution. 
	The reason for the volume dependency is the stress that arises for different microstructures at the same strain (Figure \ref{stress_lifetime_straincont}). 
	The additional soft and more ductile $\delta$-ferrite reduces the overall stress. This behavior can generally be explained as $\delta$-ferrite increases the ductility of the microstructure. Furthermore, stresses are built up first in the $\delta$-ferrite, which plastically deforms. This plastic deformation relieves stresses, affecting the whole microstructure. \\
	
	At minor loads, the more ductile ferritic phase adopts most of the strains and stresses, thus taking the loads away from the martensitic phase, which can be seen in Figure \ref{fig:mics_strain_rnd_03}. With the decreased load, the martensitic phase barely deforms. That lowered deformation leads to less plastic strain, and therefore the lifetime of the martensitic phase increases.
	The increase of ferritic volume fraction enhances the load-displacement away from the martensitic phase. In Figure \ref{subplt:mart_rnd_strain} the effect is seen by the lifetime improvement of the martensitic phase with an increase of the ferritic volume at low strains. \\
	
	
	At high loads, the $\delta$-ferrite cannot withstand the occurring high stresses. Thus the ferritic phase is significantly deforming plastically in the microstructure. By plastic deformation of the ferritic phase, stresses are relieved. With relieving of stresses, thus, for the overall microstructure, stresses are reduced depending on the volume distribution (see Figure \ref{stress_lifetime_straincont}). 
	Likewise, the load absorbed by the ferritic phase is distributed over the ferritic volume; this, in turn, means that with a small volume fraction, the ferrite is stressed relatively more. Furthermore, this explains why the lifetime is reduced for small ferritic volume fractions. 
	On the side of the martensitic phase, at high loads, almost no influence by the ferritic phase is seen (Figure \ref{subplt:mart_rnd_strain}).
	Since the ferritic load accommodation is close to its limit, part of the load is now taken up by the martensitic phase. Relatively more stresses occur in the martensitic phase, which can be seen in the microstructures of Figure \ref{fig:mics_strain_rnd}.
	Compared to the pure martensitic microstructure, the martensitic phase is stressed at a similar rate; it concludes that the lifetime of the martensitic phase at high strains achieves almost no improvement by the existence of the ferritic phase. \\
	
	Due to the fact that under strain-controlled load, different stresses depending on the volume distribution arise, stress-controlled loads are applied to the microstructure. The stresses relieved by the ferritic phase thus should be compensated. 
	Section \ref{sec:vol_rnd_str_ctr} showed that the volume dependency is reduced. Small volume dependencies on the lifetime are visible only at the lowest and highest applied stress amplitudes and otherwise nullified. The reduced volume dependency arises since the stress-controlled load counteracts the stress reduction by the ferritic phase.
	Stresses that are absorbed and relieved by the ferritic phase lead to the fact that higher strains are applied on the microstructures (see Figure \ref{plt:stress_lifetime_stresscont}). 
	From that follows, independent of the volume distribution, similar amounts of plastic strain arise in the phases. \\
	Outliers at minor stress amplitudes arise (Figure \ref{plt:phases_lifetime_stress}) since the ferritic phase is mainly responsible for the deformation and is capable of withstanding the stresses. 
	Thus at minor stress amplitudes, strains and stresses are distributed over the ferritic volume; due to this, lifetime improves by increasing the ferritic volume. \\
	
	When examining the lifetime, the question arises which phase is responsible for the crack initiation and thus the failure of the metal.
	Regarding the volume distribution, the tendency to find the crack initiation stays nearly the same under the investigated load amplitudes. In the case of the microstructure with 60HRC martensite, the crack initiation is found in the ferritic phase independent of the volume distribution.
	In the case of the microstructures with the softer martensite, the crack initiation is mainly found in the ferritic phase at high loads. At low loads, the crack tends to initiate in the martensitic phase. Nevertheless, this changes with a higher ferritic volume fraction since the loads on the martensitic phase are reduced by the ferritic phase; thus, crack initiation is mainly found in the ferritic phase. However, it is essential to note that since the lifetimes of the phases are so close to each other, the morphology significantly influences the crack initiation. 




\end{document}