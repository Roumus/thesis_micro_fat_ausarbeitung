\documentclass[../main.tex]{subfiles}
%\graphicspath{{\subfix{{../../images/}}}}
% !TeX root = ../main.tex
\begin{document}
%	\section{Analysis and Results}%Results and Discussion}
	
	\subsection{Influence of $\delta$-ferrite grain distribution on lifetime}
	\label{subsec:cluster_grains}
	
	Examining the EBSD images (Figure \ref{fig:EBSD}) of 1.4057, it can be seen that the ferritic grains cluster in columns. 
	In previous volume studies, the grains are randomly arranged in the \gls{sve}, thus they do not investigate the influence of the ferritic grain position. 
	The EBSD raises the question, if the localization of the grains affects the fatigue properties. 
	Two factors are analyzed.
	First, whether there is a load direction dependence, i.e.,  differ the results with orthogonal or parallel load in regard to the columns.
	Second, if the localization of the maximum damage shows a peculiarity or a scheme with respect to the $\delta$-ferrite column. 
	
	In order to create a \gls{sve} according to the EBSD, the grains have to be arranged in ferritic columns. The localization of the grains in the \gls{sve}  is done by defining the position of the grains in two columns (\autoref{fig:colm_examp}). 
	Those ferritic columns are generated with volume fractions of $ 5\% $, $ 10\% $, and $ 15\% $, to determine if a changed volume enhances any possible effects. Ten realizations for each of the models are generated.
	The actual investigation of the loading direction is carried out with loads orthogonal and parallel to the ferritic columns. 
	Strain amplitudes of $ 0.3\% $, $ 0.6\% $, and $ 0.9\% $ are applied. Similar to the volume study of randomly distributed grains.  
	Exemplary, a model of the microstructure is visualized in \autoref{fig:colm_examp}. \autoref{fig:mic_col_exp_ldir_whole} shows the distribution of the ferrite next to the martensite. \autoref{fig:mic_col_exp_ldir_ferr} shows the ferritic grains only.  
	
	\begin{figure}[H] 
		\centering
		\begin{subfigure}{0.48\linewidth}
			
			\includegraphics[width=\linewidth]{columnwise_example_mats.png}
			\caption{}
			\label{fig:mic_col_exp_ldir_whole}
		\end{subfigure}
		\begin{subfigure}{0.48\linewidth}
			
			\includegraphics[width=\linewidth]{columnwise_example_grains.png}
			\caption{}
			\label{fig:mic_col_exp_ldir_ferr}
		\end{subfigure}
		\caption{Example for generated microstructure, with two ferritic columns (blue and white) and the grains of the ferritic phase.
		The microstructure is cut diagonally. }
		\label{fig:colm_examp}
	\end{figure}
	
	The results regarding the directional load are examined using the observation of the scatter and the median.
	They are illustrated by boxplots, where the boxes show the $ 25\% $ to $ 75\% $ percentiles lifetimes.
	The resulting lifetimes by parallel and orthogonal loads are compared for ferritic columns with the same volume fraction (see \autoref{plt:clm_whole_box}). 
	Regardless which phase is considered, there tends to be a greater scatter with loads orthogonal to the ferritic column.
	The mean value of the determined lifetime does not differ very much between the two loading directions. It is always within the scatter range of the results. However, a small systematic effect is found and the lifetime for parallel loading is slightly larger than for orthogonal loading. 
	
	
	
	\begin{figure}[H]%{0.5\textwidth}
		\centering
		%		\hfill
		\begin{subfigure}[b]{0.45\textwidth}
			\centering
			%	\resizebox{0.6\linewidth}{!}{
			\subimport*{../images/plots/}{Whole_0p3_Life_over_Vol_Box.tex}%}
			
			\caption{$0.3\%$ strain}
			\label{subplt:clm_03_whole}
		\end{subfigure}%	
		\hfill
		\begin{subfigure}[b]{0.45\textwidth}
			\centering
			%	\resizebox{0.6\linewidth}{!}{
			\subimport*{../images/plots/}{Whole_0p6_Life_over_Vol_Box.tex}%}
		
			\caption{$0.6\%$ strain}
				\label{subplt:clm_0,6_whole}
		\end{subfigure}%
		\hfill
		\begin{subfigure}[b]{0.45\textwidth}
			\centering
			%	\resizebox{0.6\linewidth}{!}{
			\subimport*{../images/plots/}{Whole_0p9_Life_over_Vol_Box.tex}%}
			
			\caption{$0.9\%$ strain}
			\label{subplt:clm_09_whole}
		\end{subfigure}
		\hfill
		
		\caption{Lifetimes of microstructures with clustered ferritic grains in two columns. }
		\label{plt:clm_whole_box}
		\hfill
	\end{figure}
	
	In the following, the location of the damage is analyzed. It is investigated whether the damage has any particularity or scheme in relation to the $\delta$-ferrite columns.  
	As an example, \autoref{fig:mics_78_09_zz_90_col2} visualizes the results of a microstructure under a strain amplitude of $ 0.9\% $ orthogonal to the ferritic columns.
%	Independent of the load direction, other models'have similar visualizations. Thus \autoref{fig:mics_78_09_zz_90_col2} is representing other models.
	In most of the microstructure models, the greatest damage is at a triple point in a ferritic grain. The relative damage appears mainly in the ferritic phase (\autoref{subfig:mic_clm_dmg_zz}).
	Only in exceptional cases, crack initiation occurs in the martensitic phase.
	At grain boundaries of the $\delta$-ferrite grains the highest strains are accumulated (\autoref{subfig:mic_clm_strain_zz}).  Stresses that form in the martensitic phase seem not to be large enough, concerning the critical shear stress, to damage the martensite similar to the ferrite (\autoref{subfig:mic_clm_dmg_zz}). Thus, \autoref{subfig:mic_clm_dmg_zz} highlights the ferritic grains as the damaged phase. 
	
	
	
	
	\begin{figure}[H] 
		\centering
		\begin{subfigure}{0.485\linewidth}
			\def\svgwidth{\linewidth}
			\InkScapeInput{col_78_09_zz_90vol_mats.pdf_tex}
			\caption{phases}
			\label{subfig:mic_clm_mats_zz}
		\end{subfigure}
%		\begin{subfigure}{0.485\linewidth}
%			\label{subfig:mic_clm_stress_zz}
%			\def\svgwidth{\linewidth}
%			\InkScapeInput{col_78_09_zz_90vol_stress.pdf_tex}
%			\caption{stress}
%		\end{subfigure}
		\begin{subfigure}{0.485\linewidth}
			\def\svgwidth{\linewidth}
			\InkScapeInput{col_78_09_zz_90vol_strain.pdf_tex}
			\caption{strain}
			\label{subfig:mic_clm_strain_zz}
			%			 \caption{128}
		\end{subfigure}
		\begin{subfigure}{0.485\linewidth}
			\def\svgwidth{\linewidth}
			\InkScapeInput{col_78_09_zz_90vol_dmg.pdf_tex}
			\caption{relative damage}
			\label{subfig:mic_clm_dmg_zz}
			%			 \caption{128}
		\end{subfigure}
		\hfill
		\caption{Microstructure with two ferritic columns. Loaded with 0.9\% strain orthogonal to the ferritic columns.}
		\label{fig:mics_78_09_zz_90_col2}
	\end{figure}

\newpage
\end{document}