\documentclass[../main.tex]{subfiles}
%\graphicspath{{\subfix{{../../images/}}}}
% !TeX root = ../main.tex


\begin{document}

\subsection{Effect of $\delta$-ferrite grain size on lifetime}
%allgemein

In Section \ref{sec:grain_size_study} it is investigated how the ferritic grain size influences the lifetime of the microstructure.
Figure \ref{plt:KnGr_lifetime_strain_stress} shows that a different ferritic grain size significantly affects the microstructure lifetime.


The reason for the changed lifetime is the critical shear stress of the $\delta$-ferrite grains, which varies with the grain size.
This relationship between the critical shear stress and the grain size is modeled by the Hall-Petch equation (Equation\ref{eq:Hall_Petch}). %
A low critical shear stress of $52.4$ MPa is assigned to the ferritic grains with a size of $20.3 \mu m$. 
Therefore, dislocations in large $\delta$-ferrite grains are more likely to move, resulting in greater plastic deformation (compared to a smaller grain).
As a result, more plastic strains might accumulated,  which reduces the lifetime.
A small ferritic grain, in contrast, increases the critical shear stress. 
Fewer plastic strains accumulate, and thus the lifetime is improved. 
The lifetime of the microstructure does not depend on whether stress- or strain -controlled load is applied (see  Figure \ref{subplt:kg_whole_Wohler} and \ref{subplt:kg_whole_Wohler_strctr}). 
%. A small, but for the overall structure irrelevant difference exists only for the martensitic phase.

In this investigation, the martensitic grain size and the martensitic material parameters are constant. 
In the reference condition, with a ferritic grain size of $8.3 \mu m$, the $\delta$-ferrite has a lower critical shear stress than the martensite.
With the reduction of the ferritic grain size, the ferritic grains have a significantly higher critical shear stress ($287.7$ MPA) compared to the investigated 37HRC martensite ($ 181 $ MPa), which gives us a reason to observe the phases individually and compare them. 

%Figure \ref{subplt:kg_fert_Wohler} and \ref{subplt:kg_fert_Wohler_strctr} show the ferritic phase's lifetime in regard to the grain size. 
Due to the higher critical shear stress of the small ferritic grains ($4.2 \mu m$), the ferritic phase can withstand higher stresses until plastic deformation occurs. Thus at strain-controlled load in the ferritic phase stresses up to $1300$ MPa occur (Figure \ref{fig:mic_KnGr_stress_688}). Stress-controlled loads also benefit from the small grain size. Less strain has to be applied to the microstructure (Figure \ref{fig:mic_KnGr_mat_688}); thus, less plastic strain is generated. 
Interestingly, the ferrite takes with reduced grain size a bigger part of the load away from the martensite (compare Figure \ref{fig:mics_stresses_KnGr_85_dependent} and \ref{fig:mics_stresses_KnGr_688_dependent}).
 Thus, a changed ferrite grain size also affects the lifetime of the martensitic phase (Figure \ref{subplt:kg_mart_wohler} and \ref{subplt:kg_mart_Wohler_strctr}). This is mainly related to the changed local loading conditions with the altered mixed microstructure.
Nevertheless, this effect is rather small in case of a ferritic grain size of $ 4.2 \mu m $. In total, crack initiation mainly takes place in the - now softer - martensitic phase.
%%%%

In the case of the $20.3 \mu m$ ferritic grain, less stress is needed to deform the ferritic phase plastically due to the lower critical shear stress. 
The resulting plastic strains occur mainly at the ferritic grain boundaries, which have a higher magnitude with bigger ferritic grain size (compare Figure \ref{fig:mic_KnGr_stress_6} with \ref{fig:mic_KnGr_stress_688} ). 
%Thus, gives the martensitic microstructure a more homogeneously  appearance (Figure \ref{fig:mic_KnGr_stress_6}), in comparison to the other grain sizes (Figure \ref{fig:mic_KnGr_stress_85} and \ref{fig:mic_KnGr_stress_688}). 
Under stress-controlled load, a significantly higher strain must be applied to the microstructure to achieve the same stress (Figure \ref{fig:mic_KnGr_mat_6}). Those high strains must be applied because stresses are relieved by the plastic deformation of the ferritic phase.
Due to the relieve of stress by the ferritic phase, consequently, the martensic phase is more stressed to obtained the aimed stress amplitude. The higher load in the martensite reduced its lifetime. 
However, the decrease in the lifetime of the ferrite with a grai size of $20.3 \mu m$ is more significant, so that the crack initiation occurs in the ferritic grain.
The effect of large ferritic grains on ferritic lifetime is independent of the type of loading.  
On the lifetime of the martensitic phase, a distinction can be made between the type of loading. 
In contrast to stress-controlled loads, an increase in ferritic grain size from $8.4 \mu m$ to $20.3 \mu m$ leads  to later crack initiation in the martensite at strain-controlled load.
Due to the stress relief by the ferritic phase, overall, less stress occurs in the microstructure to achieve the aimed strain (Figure \ref{fig:mic_KnGr_stress_6}).  
With a grain size of $20.3 \mu m$ the general lowered stress results in less plastic strain of the martensitic phase compared to the microstructure with $8.4 \mu m$ ferritic grains. \\
The lifetimes of the martensite, with $20.3 \mu m$ and $8.4 \mu m$ ferritic grains under strain-controlled load, are relatively similar (Figure \ref{subplt:kg_mart_wohler}); thus, since the morphology can have a significant influence and the crack tends to be found in the ferritic phase, this grain size is not discussed further. 


In summary, it was possible to demonstrate that the lifetime of the material is reduced with increasing ferrite grain size and raised with decreasing grain size.
A change in the crack initiation location from the ferritic to the martensitic grains was observed, which can be attributed to the increasing strength of the ferrite with decreasing grain size.
%The location of crack initiation plays an important role for the resulting lifetime and will be discussed separately in the following section. (?)
In conclusion, the results of this study show that large, soft ferrite grains have a very detrimental effect on the lifetime of a mixed ferritic-martensitic microstructures. 

\end{document}
