\documentclass[../main.tex]{subfiles}
%\graphicspath{{\subfix{{../../images/}}}}
% !TeX root = ../main.tex


\begin{document}

\subsection{Grain size influence}
%allgemein

In Section \ref{sec:grain_size_study} it is investigated how the ferritic grain size influences the lifetime of the microstructure.
Figure \ref{plt:KnGr_lifetime_strain_stress} shows that the grain size significantly influences the whole microstructure.
A change in the ferritic grain size influences both phases.
The background of the grain size dependency is discussed below. \\

The Foundation of the investigation lies on the Hall Petch equation (\ref{eq:Hall_Petch}). The critical shear stress of the ferritic phase is linked to its grain size.
Low critical shear stress is assigned to the ferritic phase with a grain size of $20.3 \mu m$. 
The ferritic phase flows at lower loads due to the reduced critical shear stress.
Consequently, more plastic strains accumulate in the ferritic phase leading to a lower lifetime.
A small ferritic grain, in contrast, increases the critical shear stress. 
Fewer plastic strains accumulate, and thus the lifetime is improved. 
At consideration of the total lifetime the grain size is not significantly influenced by the type of load (see  Figure \ref{subplt:kg_whole_Wohler} and \ref{subplt:kg_whole_Wohler_strctr}). 

In this investigation, the martensitic grain size and thus the material parameters are constant. The reference grain size of $8.3 \mu m$ has lower critical shear stress than the martensite.
With the reduction of the ferritic grain size, the ferritic grain has significantly higher critical shear stress compared to the investigated 37HRC martensite, which gives us a reason to observe the phases individually and compare them.\\

%Figure \ref{subplt:kg_fert_Wohler} and \ref{subplt:kg_fert_Wohler_strctr} show the ferritic phase's lifetime in regard to the grain size. 
Due to the higher critical shear stress of the small ferritic grains, the ferritic phase can deform more and withstand higher stresses until plastic deformation occurs. Thus at strain-controlled load in the ferritic phase stresses up to $1300$ MPa occur (Figure \ref{fig:mic_KnGr_stress_688}). Stress-controlled loads also benefit from the small grain size. Less strain has to be applied to the microstructure (Figure \ref{fig:mic_KnGr_mat_688}); thus, less plastic strain is generated. 
The highly formable ferrite takes a large part of the load away from the martensite. Thus, the life of the martensitic phase also benefits from the small ferritic grain (Figure \ref{subplt:kg_mart_Wohler} and \ref{subplt:kg_mart_Wohler_strctr}). The effects on the martensitic phase are comparatively so small that, in the case of small ferritic grains, the martensite is mainly the crack-causing phase.\ \

In the case of the $20.3 \mu m$ ferritic grain, it takes less stress to deform the ferritic phase plastically due to the lower critical shear stress. 
Stresses in the ferritic phase are lower, which gives the microstructure a more homogeneously stressed appearance (Figure \ref{fig:mic_KnGr_stress_6}), in comparison to the other grain sizes (Figure \ref{fig:mic_KnGr_stress_85} and \ref{fig:mic_KnGr_stress_688}). 
Under stress-controlled loading, a significantly higher strain must be applied to the microstructure to achieve the same stress (Figure \ref{fig:mic_KnGr_mat_6}). The high strains must be applied because stresses are relieved by the plastic deformation of the ferritic phase.
Since the ferrite takes less load away from the martensite, consequently, the martensite is more loaded. The higher load in the martensite thus leads to more plastic strains and subsequently to a shorter lifetime. 
The effect of large ferritic grains on ferritic lifetime is independent of the type of loading. 
On the lifetime of the martensitic phase, a distinction can be made between the type of loading. 
Due to the stress relief by the ferritic phase, overall, less stress occurs in the microstructure to achieve the aimed strain (Figure \ref{fig:mic_KnGr_stress_6}).  
The lower load results in a less plastic strain of the martensitic phase compared to the microstructure with $8.4 \mu m$ ferritic grains. The Lifetimes of the microstructures, with $4.2 \mu m$ and $8.4 \mu m$ ferritic grains under strain-controlled load, are relatively close (Figure \ref{subplt:kg_mart_wohler}); thus, since the morphology can have a significant influence and the crack tends to be found in the ferritic phase, this grain size is not discussed further. 


To recap the grain size study, concluding that the lifetime of the material is reduced with increasing grain size and raised with decreasing grain size.
Depending on the grain size, crack initiation can thus also begin in the martensitic phase.
Different grain sizes exist in the analyzed material and may occur in any material point; thus, in high-stressed areas, particularly, large grains of the ferritic phase are problematic for the fatigue properties. 
Crack initiation consequently is expected in a component where high loads occur, and large grains are unfavorably oriented to other grains.\\

\end{document}
