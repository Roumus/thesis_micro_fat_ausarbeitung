\documentclass[../main.tex]{subfiles}
%\graphicspath{{\subfix{{../../images/}}}}
% !TeX root = ../main.tex


\begin{document}


\subsection{crack localization}
%%% beschreibe wo sich schlussendlich riss vorfinden lässt.
%%Eifluss aller Kapitel

%% Beschreibe wieso dieses Kapitel
In the scope of this work is the investigation of the influence of $\delta$-ferrite on the mainly martensitic 1.4057.
As discussed, the $\delta$-ferrite increases the ductility and the resulting overall stress of the microstructure. 
The critical question regarding the ferritic influence is the localization of the crack initiation; which phase is responsible for the fatigue.
In Section \ref{sec_results} in the specific cases is described where the crack might initiate. 
Nonetheless, a summary of the statements is discussed in the following.



%% Bezug zu korn cluster studie

The investigation of the ferritic grain distribution (Section \ref{subsec:cluster_grains}) shows that a cluster of ferritic grains has no significant influence. 
Since these ferritic columns are a cluster of several grains without defined orientation, the column did not represent a single anisotropic grain. 
Instead, we find a shift in the areas with ferritic influence. 
Fatigue occurs due to the inhomogeneity of the microstructure. The ferritic phase increases the inhomogeneity. However, due to the significance of the morphology, the crack is not necessarily found at the ferritic line. 
Due to the increased inhomogeneity, there is a tendency to find the crack initiation close to the ferritic column. However, unfavorable combinations of grain orientations can occur in the rest of the microstructure, leading to significantly more significant damage. 
Therefore, the position of the phases is not relevant for crack initiation, but instead, it must be determined how the $\ delta$-ferrite increases the inhomogeneity.
Consequently, the following discusses which combination of the investigated parameters maximizes the tendency for crack initiation in a specific phase.\\


Depending on the combination of load and geometrical parameters, the probability of crack initiation is shifted to one phase.\\
Crack initiation in the ferritic phase is most likely found when the microstructure is under high stress, and the $\delta$-ferrite has a small volume fraction (see Figure \ref{plt:phases_lifetime_strain} and \ref{plt:phases_lifetime_stress}). In this case, the plastic deformation in the ferritic phase is exceptionally high since the accumulated stresses limit the ferritic loading capacity (Figure \ref{fig:mics_strain_rnd}).
Large ferritic grains lowered the load capacity of  $\delta$-ferrite. This further increases the probability of crack initiation in the ferritic phase. \\

Under small loads and a grian size of $8.4 \mu m$ plastic deformation occurs in both phases to a similar extent (Figure \ref{plt:strain_single_comp_fmax}), but since the martensite can withstand less plastic deformation, it leads to earlier crack initiation (Figure \ref{plt:phases_lifetime_strain} and \ref{plt:phases_lifetime_stress}).
A small ferritic grain increases the probability of finding the crack initiation in the martensite since, in relation, the ferrites can take up significantly more load due to the high critical shear stress (Figure \ref{plt:KnGr_lifetime_strain_stress}). 
Thus the highest tendency to find the crack initiation in martensite is under small load and low ferritic volume fraction.
\\


%%allgemein beste komb gegen initie


An optimal combination can be determined from the collected information of the conducted experiments.
A high ferritic volume fraction reduces the load and thus the amount of plastic strain in the martensite and ferrite. The reduction of the ferritic grain size also leads to higher load absorption (Figure \ref{fig:mics_stresses_KnGr_688_dependent}). The reduced loads increase the martensitic phase's total lifetime; thus, the microstructure's total lifetime is improved.  
From this, it is concluded that the total lifetime of the microstructure is best with small ferritic grains and a high volume fraction of $\delta$-ferrite.


\end{document}