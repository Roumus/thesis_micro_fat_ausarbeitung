\documentclass[../main.tex]{subfiles}
%\graphicspath{{\subfix{{../../images/}}}}
% !TeX root = ../main.tex
\begin{document}
%	\section{Appendix}
%	\addcontentsline{toc}{section}{Abstract}

%Metals consist of one or more phases, which define the material's properties.
%Fatigue, which describes the failure of cyclically loaded metals, has its origin of damage in the microstructure. 
%Even in metals with a single phase, fatigue occurs due to the properties of grains of which metal consists. 
%Additional phases add further factors that influence fatigue, which this work explores.
%Depending on the microstructural properties, fatigue behavior can be approximately predicted by micromechanical modeling of a representation of the metal.  \\
%
%In this thesis, micromechanical fatigue simulations are performed to investigate fatigue crack initiation in the multiphase metal 1.4057 (X17CrNi16-2). 
%Therefore first generating representative models of the microstructure of this mainly martensitic metal, in which the grains are assigned the material properties of the individual phases.
%The micromechanical stresses in the microstructure model are determined using the phenomenological crystal plasticity \gls{CP} approach. 
%The resulting stresses and strains are used to determine the fatigue damage with the help of fatigue indicator parameters \gls{fip}, which make it possible to determine the lifetime, i.e., the number of cycles until crack initiation. \\
%
%The scope of this work is investigating the influence of the phases on fatigue properties. 
%Therefore, several models are to be investigated under different characteristics. 
%One is the study of the influence of different volume distributions of the phases under strain- and stress-controlled loading. 
%The second is to investigate the influence of the grain size of a phase on the microstructure, considering the Hall Petch equation. Since depending on the grain size, the deformability of the phase changes. \\  A change of the mentioned characteristics changes the fatigue properties and thus the resulting lifetime since, in this work, the damage is determined by the accumulation of plastic strain.
%The investigations showed that under the assumptions made, the influence of the grain size on the fatigue life is more significant than the volume fraction. \\
%However, investigations of the volume fraction showed that a further phase influences the lifetime. In the case of strain-controlled loading, lifetime is worsened by the existence of a second phase. The influence depends on the difference in the materials' properties, such as deformability and hardness.  \\
%
%With the help of these investigations, we are establishing possible causes of fatigue failure in a multiphase material. 
%Cracks in a multiphase material are likely to be found in the area of high loading and large grains of the softer, more ductile phase.   
%
%
%\newpage


	\section*{Zusammenfassung}

	Obwohl die Forschung zur Ermüdung von Stählen bis zum Jahr 1860 \cite{wohler_versuche_1860} zurückgeht, ist dies heutzutage immer noch ein bedeutendes Thema für die Industrie und die akademische Welt. Ermüdung bedeutet das ein Metall aufgrund von zyklischer Last versagt. Dabei entstehen aufgrund von Inhomogenitäten mikromechanische Verformungen in der Mikrostruktur der Metalle.   Die Mikrostruktur ist die polykristalline Struktur des Metalls. Sie besteht also aus vielen einzelnen Kristallen, die Körner genannt werden und sich durch die zugrunde liegende Orientierung des Kristalls unterscheiden. Des Weiteren unterscheiden sich Körner durch ihre chemische Zusammensetzung. Den Körnern selber chemischer Zusammensetzung entsprechen eine Phase des Materiales. Zusätzliche Phasen in der Mikrostruktur erhöhen die Inhomogenität, wodurch sich die Ermüdungseigenschaften ändern. Der Einfluss Änderung soll in dieser Arbeit untersucht werden.
	
	In dieser Arbeit werden mikromechanische Ermüdungssimulationen durchgeführt, um die Ermüdungsrissbildung in dem zweiphasigen Metall 1.4057 (X17CrNi16-2) zu untersuchen. Dazu werden mikromechanische Modelle der Mikrostruktur erstellt. Die mikromechanischen Verformungen in der Mikrostruktur werden mit dem phänomenologischen Ansatz der Kristallplastizität \gls{CP} Methode bestimmt. Die resultierenden Spannungen und Dehnungen werden zur Bestimmung der Ermüdungsschädigung mit Hilfe von fatigue indicator Parameter \gls{fip} verwendet, die es ermöglichen, die Lebensdauer, d.h. die Anzahl der Zyklen bis zur Rissinitiierung, zu bestimmen. 
	
	Um den Einfluss einer zweiten Phase auf die Lebensdauer zu untersuchen, wird in dieser Arbeit, die Volumenverteilung, die Kornverteilung und die Korngröße variiert. Diese Modellvariationen werden unter dehnungs- und spannungskontrollierten Belastungen untersucht. Durch solche Veränderungen der Mikrostruktur verändern sich die Ermüdungseigenschaften und damit die resultierenden Lebensdauern.\\
	Untersuchungen der Volumenanteils zeigten, dass eine weitere Phase die Lebensdauer beeinflusst. Die Existenz einer zweiten weichen Phase verschlechtert die Lebensdauer der Mikrostruktur. Dabei hängt der Einfluss davon ab, wie stark sich die Materialien in ihren Eigenschaften, wie Festigkeit und Härte, unterscheiden.  Durch die Korngrößenstudie ergab sich jedoch, dass unter der Berücksichtigung der Hall Petch Beziehung der Einfluss der Korngröße auf die Ermüdungslebensdauer am bedeutendsten ist. 
	


%	Metalle bestehen aus einer oder mehreren Phasen, welche die Eigenschaften des Werkstoffs bestimmen.
%	Die Ermüdung, welche das Versagen von zyklisch belasteten Metallen beschreibt, hat ihren Ursprung in der Mikrostruktur. 
%	Selbst bei Metallen mit einer einzigen Phase tritt Ermüdung auf, aufgrund der Eigenschaften der Körner, aus denen das Metall besteht. 
%	Durch das Hinzufügung anderer Phasen kommen weitere Faktoren hinzu, welche die Ermüdung beeinflussen, was in dieser Arbeit untersucht wird. 
%	Abhängig von den mikrostrukturellen Eigenschaften kann das Ermüdungsverhalten durch mikromechanische Modellierung einer representation des Metalls annähernd vorhergesagt werden.  \\
%	
%	In dieser Arbeit werden mikromechanische Ermüdungssimulationen durchgeführt, um die Ermüdungsrissbildung in dem zweiphasigen metall 1.4057 (X17CrNi16-2) zu untersuchen. 
%	Dazu werden zunächst repräsentative Modelle des Gefüges, dieses überwiegend martensitischen Metalls, erstellt, in denen den Körnern die Materialeigenschaften der einzelnen Phasen zugeordnet werden.
%	Die mikromechanischen Spannungen im Gefügemodell werden mit dem phänomenologischen Ansatz der Kristallplastizität \gls{CP} bestimmt. 
%	Die resultierenden Spannungen und Dehnungen werden zur Bestimmung der Ermüdungsschädigung mit Hilfe von Ermüdungsindikatoren \gls{fip} verwendet, die es ermöglichen, die Lebensdauer, d.h. die Anzahl der Zyklen bis zur Rissinitiierung, zu bestimmen. \\
%	
%	Im Rahmen dieser Arbeit soll der Einfluss der Phasen auf die Ermüdungseigenschaften untersucht werden. 
%	Daher sollen mehrere Modelle mit unterschiedlichen Eigenschaften untersucht werden. 
%	Zum einen soll der Einfluss unterschiedlicher Volumenverteilungen der Phasen unter dehnungs- und spannungskontrollierten Belastung untersucht werden. 
%	Zum anderen soll der Einfluss der Korngröße einer Phase auf das Gefüge, unter Berücksichtigung der Hall-Petch-Gleichung, untersucht werden. Dadurch ändert sich in Abhängigkeit von der Korngröße die Verformbarkeit der Phase. \\ Eine Änderung der gennanten Eigenschaften verändert die Ermüdungseigenschaften und damit die resultierende Lebensdauer, da in dieser Arbeit die Schädigung durch die Akkumulation der plastischen Dehnung bestimmt wird.
%	Die Untersuchungen zeigten, dass unter den getroffenen Annahmen der Einfluss der Korngröße auf die Ermüdungslebensdauer bedeutender ist als der Volumenanteil. \\
%	Untersuchungen des Volumenanteils zeigten jedoch, dass eine weitere Phase die Lebensdauer beeinflusst. Bei dehnungskontrollierter Belastung wird die Lebensdauer durch die Existenz einer zweiten Phase verschlechtert. Dabei hängt der Einfluss davon ab, wie stark sich die Materialien in ihren Eigenschaften, wie Verformbarkeit und Härte, unterscheiden.  \\
%	
%	Mit Hilfe dieser Untersuchungen ermitteln wir mögliche Ursachen für Ermüdungsversagen in einem mehrphasigen Werkstoff. 
%	Risse in einem mehrphasigen Werkstoff sind wahrscheinlich im Bereich hohen Belastung und großen Körner der weicheren, duktilen Phase zu finden.   

\end{document}